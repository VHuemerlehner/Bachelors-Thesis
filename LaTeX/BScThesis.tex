%Quick Build F1
%BibTex F11

\documentclass[a4paper,12pt,twoside]{report}
\usepackage{amsmath}
\usepackage[]{graphicx}
\usepackage[utf8]{inputenc}
\usepackage{amssymb}
\usepackage{CogSciBScUOS}


\title{Automatic Extraction of Harmonic Rhythm from Melodies}
\author{Valentin Huemerlehner}
\email{vhuemerlehne@uni-osnabrueck.de}
\firstSupervisor{Prof. Dr. Kai-Uwe Kühnberger}
\secondSupervisor{PhD Maximos Kaliakatsos-Papakostas}


\begin{document}

\beforepreface %needed clause, generates e.g. title page

\prefacesection{Abstract}
In this thesis, I tried to extract the harmonic rhythm from a given melody. Alongside the melody, information on
cadence placement and the bar boundaries are given. The system learned style-specific patterns from an
annotated corpus of pieces in different styles, with the same information as above and additionally the original
harmonisation in a reduction of medium depth. With this, I wanted to answer the question whether or not it may
be possible to define one of the two elements necessary for a harmonisation (the harmonies and their locations)
independently of the other. Achieving this would mean a step forward towards an autonomous harmonisation tool
that is most importantly independent of its user's expert knowledge. Thus, such a system could also be used by
musical beginners or completely musically inept users.

\prefacesection{Acknowledgements}
I would like to thank my supervisor Kai-Uwe Kühnberger, who gave me the idea and encouragement to try a
project within musicology and then put me in touch with the musicology lab of Thessaloniki. There, I owe thanks
to Emilios Cambouropoulos and Tsougras Costas for their musicological as well as computational insights and
advice. However, the most help I received from Maximos Kaliakatsos-Papakostas, who had available not only
advice, but also almonds in great abundance. His Matlab skills proved indispensable at many times and all ideas
I had ran through his mind as a filter of feasability.

\afterpreface %needed clause, generates e.g. table of contents
\newpage

%\listoffigures
%\newpage

\pagenumbering{arabic}

%use   \newsection{XYZ}

%instead of the "\section" command!

\chapter{Theoretical Background}
In the last years, scientific progress on computational creativity has accelerated more and more in all aspects
of creativity. In the fine arts, this progress is mainly driven by neural networks, allowing for new techniques such
as style overlaying or style blending and recombination of different aspects of an image (such as form and texture)
[citation needed]. In music, neural networks are also beginning to make their mark, one example being "DeepBach",
a neural network able to copy Bach's chorale style (or any other, provided the correct input) and already fooling
even musical experts from time to time\cite{hadjeres2016deepbach}.
Examples for systems: Flow Machines (Sony France), Watson Beat (IBM), Jukedeck (London Startup), CHAMELEON,
Songsmith (Google), Magenta (Google)

\chapter{The CHAMELEON Harmonising System}


%END normal LaTex document
\newpage
\bibliographystyle{alpha}
\bibliography{biblio}

\closing
\end{document}