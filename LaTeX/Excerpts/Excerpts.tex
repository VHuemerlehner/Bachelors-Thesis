\documentclass[a4paper,12pt]{report}
\usepackage{color}

\begin{document}

\chapter{Books}

\section{Harmony - Piston}
All of the following is summarised from chapter 12  ``Harmonic Rhythm'' in the book ``Harmony'' by
Walter Piston \cite{piston1978harmony}.\\

\subsection{Root changes as harmonic rhythm}
Piston defines harmonic rhythm as root changes (in roman numerals), thus implying an already existing
analysis of the piece in the western tonal system. He therefore views harmonic rhythm as possibly, but
not necessarily different from melodic rhythms (e.g. homophonic movement often leads to parallel
harmonic and melodic rhythms).

\subsection{Frequency of root change}
One important aspect to him is the frequency of root change, e.g. very slow harmonic rhythm in an
introductory section as opposed to restless fast changes, almost impossible to grasp for the listener.
``Comparable to static harmony is the effect of a [...] pedal [...]. [It] tends to overpower the sense
of harmonic progression.'' (p.196)

\subsection{Strength of harmonic progressions}
Depending on the harmonies involved (among other things), a harmonic progression can be more or
less ``strong'', i.e. compelling. This implies a hierarchy withing harmonic rhythm, i.e. leaving out
certain chord changes should have a higher perceptual impact than other progressions.

\subsection{Dynamic indications}
Dynamic indications can show a composer's specific idea, especially where they do not conincide with
the interpreter's intuition.

\subsection{Nonharmonic chords}
Piston further suggests the existence of nonharmonic chords (just as there are nonharmonic notes), such
that even though a root change can be observed, the chords need not necessarily be processed as
entire harmonic entities by the listener. This mainly happens in fast passenges and is facilitated by the
chords not being in root position, thus omitting the bass note and hindering easy recognition of the
chord.


\section{Harmonic Rhythm - Swain}

All of the following is summarised from the first part (chapters 2-7) in the book ``Harmonic Rhythm -
Analysis and Interpretation'' by Joseph Swain \cite{swain2002harmonic}.\\

Swain claims that his work is solely an extension of Piston's basic idea of assigning harmonic rhythm due
to root changes with contextdependent elements (speed of the piece and harmonic progression, inversion,
dynamic indications by the composer). He proposes a hierarchical system of multiple levels of harmonic
rhythm.

\subsection{Texture of the rhythm}
His first level of analysis is the texture of a piece. However, since it is not very relevant perceptually (and
is only the addition of all rhythmical events of a piece, thus giving the division into as many time spans as
possible), I mostly disregard it in this thesis which aims to give a perceptually meaningful harmonic rhythm.

\subsection{Phenomenal harmonic rhythm}
The next level is called ``phenomenal harmonic rhythm''. It denotes all time points at which the chord
changes in any way. This includes new harmonies as well as inversions of chords. However, he defines a
texture phenomenal harmonic rhythm vs a contrapuntal phenomenal harmonic rhythm. The latter ignores
very fast arpeggios and the like where the notes do not appear as single events, but instead only serve as
necessary parts of the harmony invoked (example Chopin op. 25 no. 1). To cite Swain: ``This is
contrapuntal phenomenal harmonic rhythm: the changes in harmonic phenomena caused by moving voices.''
According to Swain, phenomenal harmonic rhythm serves two purposes: A comparison with the texture
can already produce insight into the emotional valence of a passage (example Vivaldi Winter 1st movement:
moderately fast texture, very slow harmonic rhythm, creating a tension from this contrast). Secondly, when
dealing with non-tonal music (be it non-Western, modal or posttonal), this is the only level where chords
are not assigned a value or ordered in some kind of hierarchy, thus allowing for a neutral juxtaposition.

\subsection{Bass pitch harmonic rhythm}
The next higher level (sometimes coinciding with contrapuntal phenomenal harmonic rhythm) depends on
the movement of the bass voice (notably not necessarily the lowest voice in the score). This can also lead
to a higher bass rhythm than the root rhythm, when the bass leaps an octave. It may also be slower than
the root rhythm, since the same note may be used consecutively in different functions (though this raises
the question of microtonal differences in non-equal temperaments). Bass pitches were ignored in the given
analyses that I used, therefore, I will not lay any more focus onto it than this.

\subsection{Root/Quality harmonic rhythm}
With this level of harmonic rhythm analysis, Swain returns to Piston's original idea of harmonic rhythm
being mainly defined by chord changes. However, he adapts the system in two important ways: First of all,
he does not use a Roman numeral analysis as the basis of his rhythm analysis, but instead uses the
absolute chord symbols (e.g. D instead of I, A instead of V etc.). He wants to extract the information more
precisely by doing so (Functional harmonic rhythm will appear lateron). Also, it allows for ambiguity on the
root level, while on the functional level, one will have to decide for one of multiple possible analyses.
The second, probably more remarkable change in comparison to Piston's approach is to use hierarchical
levels of root harmonic rhythm: On the lowest level, all chord changes are taken into consideration, while
on higher levels, contrapuntal coincidences that can be subsumed into a longer lasting primary triad are
omitted.\footnote{This idea of hierarchical harmonic analysis seems strongly influenced by Schenker,
whose ideas also build the foundation of the work of Prof. Martin Rohrmeier, TU Dresden.} This type of
analysis obviously requires fine judgement of a variety of factors, because there are decisions to be made
about the number of levels needed and also the exact shape each of this levels is supposed to take. Swain
formulates a number of rules to achieve a meaningful analysis at this level that I will not present in further
detail here, but may be of interest to the more musicologically inclined reader.

\subsection{Density of harmonic rhythm}
On another (though not necessarily more complex) level, Swain considers the chords' density and thus,
the strength of the harmonic progresison. The density of a progression is defined by the number of voices
that change their sounding note into one that is part of the new harmony. While a singular density may
not be of high perceptual importance, a passage with high density values will tend to appear more
powerful than one with very low density changes, even if they may be of the same dynamic force. Again,
Swain goes into more detail of the interpretational possibilities and challenges (such as two voices leading
into the same note or even the question how to define a voice e.g. in a piano score), that I will not cover
here.

\subsection{Rhythm of harmonic functions}
The last level of analysis is that of harmonic functions. While the original idea of Piston used them, the
root level probably describes his intentions better. As does Piston, Swain allows for embedding of short
phrases that are not long or significant enough to be called an entire modulation. Swain however only
uses three symbols (I, IV and V for tonic, subdominant and dominant) since these three basic functions
should suffice in his opinion. Thus, he has to use embedding more often than Piston.

\subsection{Thoughts and Remarks}
While almost all of the levels mentioned above seem perceptually important, it is impossible to model
them all within the span of this thesis and I will have to focus on one level. It seems intuitive to choose an
intermediate level for this purpose so the analysis does not lose itself in too much detail or stays at an
overly broad level. This excludes texture and phenomenal harmonic rhythm (at least the textural kind).
Density seems to lack semantic meaning, but may be an interesting addition to check for a transition's
importance. This leaves bass, root and functional harmonic rhythm: Functional rhythm might, just as
density, be interesting to check in terms of a transition's importance (e.g. V to I in a cadence seems
very salient), but is a concept that is very hard to define precisely, especially since I worked with already
existing analyses of the corpus that worked with root chords. The bass is certainly interesting, but seems
mostly subsumed under root changes. Finally, within the root rhythm category, we have different
hierarchical levels at our disposal. Again, I worked with given analyses, so I did not have much of a choice,
but most of the time, the analyses seemed to interpret the pieces on a somewhat intermediate level,
which appears desirable from a complexity standpoint.


\section{Computational Music Analysis - Meredith}
\cite{meredith2016computational}
In the second chapter of this book edited by David Meredith, Emilios Cambouropoulos goes into detail on
the GCT (General Chord Type) representation of chords that is being used in the CHAMELEON harmoniser
as well.

\section{Music and Probability - Temperley}
\cite{temperley2007music}
In his book ``Music and Probability'', Temperley puts forth his earlier formulated idea of \emph{
communicative pressure} \cite{temperley2004communicative}, and uses different examples to show his
idea: since music has the goal of communicating between composer/interpreter and listener, a common
ground of this communication, some kind of ruleset, needs to be set so that the listener has any chance
to understand the composer's intentions. This includes style-specific amounts of syncopation and rubato
to be used. Temperley notes that styles with high syncopation tend to be very strict in the tempo used and
vice versa. This supports the idea that concurring factors of complexity in music may be negatively
correlated. Thus, it seems reasonable to assume that this might also apply to melodic rhythm and
harmonic rhythm in terms of speed (since processing speed of the brain is limited) and complexity.

\chapter{Articles}

\section{Harmonic Rhythm in Beethoven's symphonies - LaRue}
\cite{la2001harmonic}
In his article in the ``Journal of Musicology'', LaRue describes in length his interpretation as to why
Beethoven's symphonies seem to be widely accepted as masterpieces of orchestral music. I am not so
much interested in that than more in the way he defines and analyses harmonic rhythm. His article was
published a year before Swain's book on the topic appeared, so his work seems to be based more on
Piston's notion of harmonic rhythm. This means that LaRue is only interested in root changes, drawing
from an analysis of functions denoted as Roman numerals. Other dimensions are mentioned, but do not
find higher interest. He recognizes the problem of ambiguity, but himself does not offer a solution.

\section{DeepBach - Hadjeres and Pachet}
\cite{hadjeres2016deepbach}
Very notable is a paper yet unpublished in
journals, though already available online. The DeepBach project by Sony France is an
attempt at using Artificial Neural Networks (ANNs) for automatic feature extraction and consecutively
using the newly learned stylistic information for generation of new harmonisations in the same style. As
hinted by the title, the subject of analysis were pieces by Johann Sebastian Bach, more precisely
chorales. The entire chorale corpus was used to gain as much information as possible. To assess the
quality of the reharmonised chorale melodies, two tests were performed. 1600 subjects (among them
400 music students or professional musicians) were asked which of two versions of the same melody
fragment they liked best. DeepBach outperformed other Machine Learning algorithms here. In a second
test, roughly 1300 subjects were given the binary choice of ``Bach'' or ``Computer'' upon hearing an
extract. Again, DeepBach outperformed other tools and even managed to fool roughly 50\% of all
listeners. This shows the feasibility of ANNs in music generation, but also demonstrates the problem that
no theoretical understanding of the underlying structure is necessary and thus not generated. Therefore,
the analysis of substructures - such as the harmonic rhythm - remains a fruitful and meaningful task.

\section{Melodic accent - Thomassen}
\cite{thomassen1982melodic}
In the admittedly a little dated paper ``Melodic accent'', Joseph Thomassen conducted a series of
experiments to find out how the melodic contour influences perceived accentuation. His results are only
of interest to this thesis since he found that melodic contour is indeed one factor for which to account.
Hence, it seemed like a reasonable to idea to check for a possible relationship between contour changes
and harmonic events, providing one plausible candidate of a predictor for harmonic rhythm in the melody.

\section{Rhythm pattern perception in music - Dawe}
\cite{dawe1993rhythm}
In his dissertation, Lloyd Dawe finds experimental support for notions that music theorists have supported
over a long time (starting with Jean-Philippe Rameau). The most important aspects for this thesis are:
Chord changes are the most significant aspect for rhythm perception among harmonic
rhythm, perceptual rhythm of chords and temporal accents. Also, they seem to be interpreted as
indicators for meter, hinting at a strong prevalence of the first beat of a measure for a chord
change. Additionally, this makes the hypothesis very plausible that harmonic rhythm tends to be rather
regular.

%Not necessary!
\section{Autocorrelation in meter induction - Toiviainen, Eerola}
\cite{toiviainen2006autocorrelation}
This study is only partially relevant to my thesis: Under the assumption that harmonic rhythm somewhat
depends on the meter that seems reasonable following Dawe \cite{dawe1993rhythm}, indicators for
meter seem to be one possible candidate for being predictors for harmonic change as well.

\section{Learning and blending harmonies in the context of a melodic harmonisastion assistant -
Kaliakatsos-Papakostas, Makris, Zacharakis, Tsougras, Cambouropoulos}
\cite{kaliakatsos2016overview}
This paper lies at the heart of this thesis. It provides an overview over the by now so-called \textit{
CHAMELEON} harmonising system that is capable of learning idiom-specific style information to then
harmonise a given melody of the same or another style. The way it harmonises is by taking the user's
input as to where it should put a harmony (e.g. on a cadence) and then applying the learned ``rules'' to
this note and the preceding chords. Here is where this thesis comes into play: For non-experts, making
``good'' decisions on where to put harmonies might be hard, so giving the system the ability to analyse
the melody in such a way as to find spots where to put harmonies seems like the logical next step.

\section{Learning and Creating Novel Harmonies in Diverse Musical Idioms - Kaliakatsos-Papakostas,
Makris, Tsougras, Cambouropoulos}
\cite{kaliakatsos2016learning}
More detail about the above.

\section{Mathematical measures of syncopation - Gómez, Melvin, Rappaport, Toussaint}
\cite{gomez2005mathematical}
From this proceedings report and the idea of ``Communicative Pressure'' (see e.g. chapter 10 of \cite{
temperley2007music} stems the thought that harmonic rhythm and melodic rhythm or rather their
complexities and/or speeds may be negatively correlated, i.e. if the melodic rhythm is very complex,
composers tend to keep the harmonic rhythm at a more basal level so as to not overload the listener
with too much information. As it is very hard to properly define ``complexity'' of a melody, I instead
used syncopation as an indicator of rhythmic complexity (see \cite{fitch2007perception} for
experimental evidence). To assess the speed of a melody, I simply summed up the number of onsets
within a measure. This has the obvious shortcoming of treating a range of different rhythms and tempi
as equivalent, but was the only feasible option and should suffice for my purpose.
\textbf{\textit{\textcolor{red}{ADD EXAMPLES! (Mendelssohn Lied ohne Worte oder Hochzeitsmarsch
Intro vs. vier Viertel (think of something)}}}

\section{Perception and Production of Syncopated Rhythms - Fitch, Rosenfeld}
\cite{fitch2007perception}
In their study, Fitch and Rosenfeld tested if there was to find a correlation between the degree of
syncopation of a passage and the difficulty subjects would have with tapping a beat ``against'' the
syncopated beat, tapping a syncopated passage to a computer generated beat and remembering it
lateron. As expected, in all categories, the more syncopated a passage was, the harder it was to
fulfill any of the given tasks. This provides evidence for syncopation being a factor in making a melody
complex.



\newpage
\bibliographystyle{alpha}
\bibliography{biblio}

\end{document}